\section*{3. The Virial Theorem}

a) In space there are many gravitationally bound systems. If a system is roughly in equilibrium, the 
Virial Theorem states that the kinetic energy is equal to minus one half the potential energy 
$(\langle T \rangle = -\frac{1}{2} \langle V \rangle)$. Consider a light particle in a circular orbit
around a heavier one. Prove the Virial Theorem from this system's equation of motion.\\
\\
When we consider a light particle in a circular orbit around a havier one. Then the light has the mass $m$
and the heavier one has the mass $M$. The light orbits around the object at the radius $R$. Then the 
potential energy is:
\begin{equation*}
    V = -\frac{GmM}{R}
\end{equation*}
To figure out the kinetic energy we remeber that the gravitational force is:
\begin{equation*}
    F_{grav} = -\frac{GmM}{R^2}
\end{equation*}
while the centrifugal force is:
\begin{equation*}
    F_{cent} = \frac{mv^2}{R}
\end{equation*}
In a circular orbit these counteract each other perfectly, so we must have:
\begin{equation*}
    \frac{mv^2}{R} = \frac{GmM}{R^2}
\end{equation*}
Thus the kinetic energy of the light particle is:
\begin{equation*}
    T = \frac{mv^2}{2} = \frac{GmM}{2R}
\end{equation*}
while the kinetic energy of the heavier one is negligible, putting the previous equations in perspective
we receive:
\begin{equation*}
    \begin{split}
        V &= -\frac{GmM}{R}\\
        -V &= \frac{GmM}{R}\\
        T &= \frac{GmM}{2R}\\
        2T &= \frac{GmM}{R}\\
        2T &= -V\\
        T &= -\frac{1}{2}V
    \end{split}
\end{equation*}
