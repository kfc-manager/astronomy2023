\section*{4. Luminosity}

a) What do you understand by irradiance? Calculate the irradiance received from the Sun above the 
absorbing atmospheres of planet Mercury, Earth, and Uranus. Given, luminosity of the Sun 
$L_\odot = 3.839 \times 10^{26}W$.\\
\\
Irradiance is the power of the electromagnetic energy received by a surface. Given from the lecture we
have the following equation for power intake of a planet:
\begin{equation*}
    L_{in} = (1 - A) \frac{L_\odot \pi R^2_P}{4 \pi d^2_P}
\end{equation*}
With that equation we receive the power intake of the whole atmosphere of the planet. If we devide it now
by the area of its atmosphere we get the irradiance or power intake per area unit:
\begin{equation*}
    \begin{split}
        E_e &= \frac{L_{in}}{A_P}\\
            &= (1 - A) \frac{L_\odot \pi R^2_P}{4 \pi d^2_P} \cdot \frac{1}{\pi R^2_P}\\
            &= (1 - A) \frac{L_\odot}{4 \pi d^2_P}\\
    \end{split}
\end{equation*}
We take a look at the albedos and distance Sun to planet: 
\begin{center}
    \begin{tabular}{c|c|c}
        Planet & Albedo & Distance to Sun\\
        \hline
        Mercury & 0.06 & $0.307499 AU$\\
        \hline
        Earth & 0.31 & $1 AU$\\
        \hline
        Uranus & 0.66 & $19.201 AU$\\
    \end{tabular}
\end{center}
\textbf{Mercury:}
\begin{equation*}
    \begin{split}
        E_e &= (1 - 0.06) \frac{3.839 \times 10^{26}W}{4 \pi 0.307499^2 AU^2}\\
            &\approx (1 - 0.06) \frac{3.839 \times 10^{26}W}{4 \pi 0.307499^2 \cdot (1.496 \times 10^{11})^2 m^2}\\
            &\approx 13571 \frac{W}{m^2}
    \end{split}
\end{equation*}
\textbf{Earth:}
\begin{equation*}
    \begin{split}
        E_e &= (1 - 0.31) \frac{3.839 \times 10^{26}W}{4 \pi 1^2 AU^2}\\
            &\approx (1 - 0.31) \frac{3.839 \times 10^{26}W}{4 \pi 1^2 \cdot (1.496 \times 10^{11})^2 m^2}\\
            &\approx 941.19037 \frac{W}{m^2}
    \end{split}
\end{equation*}
\textbf{Uranus:}
\begin{equation*}
    \begin{split}
        E_e &= (1 - 0.66) \frac{3.839 \times 10^{26}W}{4 \pi 19.201^2 AU^2}\\
            &\approx (1 - 0.66) \frac{3.839 \times 10^{26}W}{4 \pi 19.201^2 \cdot (1.496 \times 10^{11})^2 m^2}\\
            &\approx 1.259 \frac{W}{m^2}
    \end{split}
\end{equation*}
b) The irradiance received by the Sun above the Earth's atmosphere per unit area, is also known as the
"solar irradiance". Find the distance from where a 60-Watt lightbulb has its irradiance equal to the solar
irradiance.\\
\\
The Eath has a solar irradiance of about $941.19037 \frac{W}{m^2}$. With that in mind we receive the 
following statement:
\begin{equation*}
    941.19037 \frac{W}{m^2} = \frac{60 W}{d_P^2}
\end{equation*}
With $d_P$ being the distance from lightbulb to Earth.
\begin{equation*}
    \begin{split}
        d_P^2 &= \frac{60 W}{941.19037 \frac{W}{m^2}}\\
              &= \sqrt{60 W \cdot \frac{1}{941.19037 \frac{W}{m^2}}}\\
              &= \sqrt{60 W \cdot \frac{1}{941.19037 \frac{W}{m^2}}}\\
              &= \sqrt{60 W \cdot \frac{1}{941.19037} \cdot \frac{1}{\frac{W}{m^2}}}\\
              &= \sqrt{60 W \cdot \frac{1}{941.19037} \cdot \frac{m^2}{W}}\\
              &= \sqrt{60 \cdot \frac{1}{941.19037} \cdot m^2}\\
              &= \sqrt{\frac{60}{941.19037}}m\\
              &\approx 0.25248m
    \end{split}
\end{equation*}
The lightbulb must be around 0.25248 metre above the atmosphere of the Earth to cause the same irradiance
as the Sun to Earth.\\
\\
c) The energy emitted per second by a star is $L = 4 \pi R^2 \sigma T^4_{eff} = S \sigma T^4_{eff}$, where
$S$ is the surface area of the star. A person also emits radiation. Under normal conditions, the 
temperature of the human body is around $37 \degree C$ or $T \approx 310 K$. An area of a person's body
is on the order of $S \approx 1.7 m^2$. What is the energy emitted per second by a person and what is the
characteristic wavelength of this emission?\\
\\
The constant $\sigma$ is here the Stefan-Boltzmann constant, which is about 
$5.67 \times 10^{-8} \frac{W}{m^2K^4}$. We can then insert the values given in the problem setting:
\begin{equation*}
    \begin{split}
        L &= S \sigma T^4_{eff}\\
          &= 1.7 m^2 \cdot 5.67 \times 10^{-8} \frac{W}{m^2K^4} \cdot 310^4 K^4\\
          &= 1.7 \cdot 5.67 \times 10^{-8} W \cdot 310^4\\
          &= 1.7 \cdot 5.67 \times 10^{-8} W \cdot 9.23521 \times 10^9\\
          &= 1.7 \cdot 5.67 W \cdot 9.23521 \cdot 10\\
          &= 890.2 W
    \end{split}
\end{equation*}
Since realistically a person can't be perfectly still, we assume that the person is walking at walking
speed. Meanin that the person would have a velocity of $1.42 \frac{m}{s}$. Then we get the characteristic
wavelength with the following equation:
\begin{equation*}
    \lambda = \frac{h}{p}
\end{equation*}
where $h$ is the Planck constant of about $6.626 \times 10^{-34} \frac{m^2 kg}{s}$. And $p$ being the
momentum of the person:
\begin{equation*}
    p = m \cdot v
\end{equation*}
An average person ways about $80 kg$. Then we receive:
\begin{equation*}
    \begin{split}
        \lambda &= \frac{h}{m \cdot v}\\
                &= 6.626 \times 10^{-34} \frac{m^2 kg}{s} \cdot \frac{1}{80 kg \cdot 1.42 \frac{m}{s}}\\
                &= 6.626 \times 10^{-34} m \cdot \frac{1}{80 \cdot 1.42}\\
                &= 5.83275 \times 10^{-36} m
    \end{split}
\end{equation*}
