\section*{3. Resolution and the Rayleigh Criterion}

The Rayleigh criterion gives the smallest possible angle between point sources or the best obtainable
resolution. Therefore the Rayleigh Criterion allows to determine the best possible resolution of a 
telescope. This criterion for the minimum resolvable angle is $\Theta = 1.22 \frac{\lambda}{D}$, with 
$\lambda$ being the wavelength of light and $D$ the opening diameter of an optional instrument.\\
\\
a) Using the Rayleigh criterion, estimate the angular resolution limit of the human eye at the wavelength
of 550 nm . Assume the diameter of the pupil is 5 mm.\\
\\
b) Compare your answer in part a) to the angular diameters of the moon and Jupiter. Can we resolve 
features on the Moon and Jupiter with just our eyes? Use your calculations to support your argument.\\
\\
c) What is the angular resolution of the radio telescope in Effenberg ($D = 100$ m, $\lambda = 21$ cm)?
Can you use this telescope to resolve features on the moon?\\
\\
