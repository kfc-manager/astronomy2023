\section*{1. Coordinate system}

a) Define \textbf{zenith}, \textbf{nadir}, the \textbf{celestial north} and \textbf{south poles}, and the
\textbf{meridian line}. Draw the meridian plane of an observer, including the position of the observer 
(*), the zenith (Z), the meridian line, and the north (N) and south (S) poles.\\
\\
\begin{itemize}
    \item \textbf{zenith}: Point which is directly over the observer on a celestial sphere
    \item \textbf{nadir}: Point which is directly under the observer on a celestial sphere
    \item \textbf{celestial north pole}: Northern point of the Earth's rotation axis
    \item \textbf{celestial south pole}: Southern point of the Earth's rotation axis
    \item \textbf{meridian line}: Great circle that passes through the celestial poles
\end{itemize}

\noindent\makebox[\textwidth]{\includegraphics[scale=0.35]{meridian_plane.jpeg}}
\noindent
b) An observer located in the Earth's Northern Hemisphere observers the top and bottom culminations of
circumpolar star. Measuring $h_i = 20 \degree \: 22' \: 32.4''$ ; $A_i = 180 \degree$ for the hight and
Azimuth of the bottom culmination and $h_s = 50 \degree \: 23' \: 08.2''$ ; $A_s = 180 \degree$ for the
upper culmination. What is the observer's latitude $\phi$?\\
\\
\noindent
c) Determine the maximum height in the sky that the globular cluster $\omega$Cen (declination 
$\delta = -47 \degree \: 29'$) reaches when observed from the Inter-American Observatory of Cerro Tololo,
Chile (latitude $\phi = -30 \degree \: 10' \: 20.9''$)\\
\\
