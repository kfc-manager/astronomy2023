\section*{4. Parallax}

a) Sirius is the biggest star in the night sky. It is located at a distance of 8.61 light years from the
Sun. Calculate it's trigonometric parallax in arcminutes.\\
\\
The radius of the Sun is about 696340 km, which would be about 7.36031961e-8 light years. With that we
can calculate the trigonometric parallax.
\begin{equation*}
    \begin{split}
        p &= arctan \biggl(\frac{\text{counter cathetus}}{\text{attatched cathetus}}\biggr)\\
          &= arctan \biggl(\frac{7.36031961 \cdot 10^{-8}}{8.61}\biggr)\\
          &\approx 4.898 \cdot 10^{-7} \degree\\
          &= 0.000029388'\\
    \end{split}
\end{equation*}
\noindent
b) Do the same for Proxima Centauri, the nearest star to the sun, which is located at a distance of 4.2
light years.\\
\begin{equation*}
    \begin{split}
        p &= arctan \biggl(\frac{\text{counter cathetus}}{\text{attatched cathetus}}\biggr)\\
          &= arctan \biggl(\frac{7.36031961 \cdot 10^{-8}}{4.2}\biggr)\\
          &\approx 1.004 \cdot 10^{-6} \degree\\
          &= 0.00006024'\\
    \end{split}
\end{equation*}
\noindent
c) Compare both to the angular size of the moon.\\
\\
The angular size of the diameter of the moon from earth is about 31.2 arcminutes. The parallax of Sirius
would be about $\frac{1}{1061642}$ of the angular size of the moon. The parallax of Proxima Centauri
would be about $\frac{1}{517928}$ of the angular size of the moon.
