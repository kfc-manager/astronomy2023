\subsection*{2. Greek Astronomy}

a) In 265 BC, the Greek astronomer Aristarch attempted to measure the distance between the sun and the
earth. To do this, he measured the angle between imaginary lines connecting the earth with the sun and the
earth with the moon to be $87 \degree$ at exactly half moon. Using his measurement, by how much is the
distance to the sun larger than the distance to the moon? Using today's values for these distances, is his
measurement accurate and in the case that it is not, what would be the angle he should have measured?\\
\\
When we see a half moon from earth that means that the light of the sun is shining perpendicular on the
line drawn from earth to the moon. In that case the angle between from moon to earth and moon to sun is
exactly $90 \degree$. Measuring now the angle between earth to moon and earth to sun would allow to 
calculate how much bigger the sun is than the moon. Aristarch measured an angle of $87 \degree$ which
would give us $\frac{1}{cos(87)} \approx 19.1073$. Meaning the sun would be about 19.1073 times bigger
than the moon. In reality the sun is about 400 times bigger than the moon.
\begin{equation*}
    \begin{split}
        400 &= \frac{1}{cos(x)}\\
        400 \cdot cos(x) &= 1\\
        cos(x) &= \frac{1}{400}\\
        x &= arccos(\frac{1}{400})\\
        &\approx 89.86 \degree\\
    \end{split}
\end{equation*}
Instead of the angle $87 \degree$ he should have measured $89.86 \degree$ to come up with a somewhat
accurate result.\\
\\
\noindent
b) Around 220 BC Eratosthenes examined the fact that there is only one day each year in the city Cyrene,
when the sun can reach the bottom of a deep well (meaning that the sun is located in the zenith), while
this is never the case in Alexandria (which is further north and located 770 km from Cyrene). 
Additionally, he found that on this day, the angle between the zenith and the sun is $7.2 \degree$ in
Alexandria. Use this measurement to infer the radius of the earth. How well does the radius compare to
today's value?\\
\\
If we imagine the earth in two dimensional space we would have a circle. Moving around that circle once
would represent $360 \degree$. If we move $7.2 \degree$ that would mean we moved 
$\frac{7.2 \degree}{360 \degree} = 0.02$ times around that circle. If the bit we moved would be 770 km
that would mean that the whole circle have a circumference of 
$50 \cdot 770 \text{ km} = 38500 \text{ km}$ with this calculation and given parameters. In reality
earth's circumference is about 40030 km. The calculation is actually close to an accurate result for that 
time.
